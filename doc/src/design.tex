\documentclass[11pt]{article}

\usepackage{amsmath}
\usepackage{graphicx}

\title{Junta\\Build management tool}
\author{Peter Penzin}
\date{\today}

\begin{document}

\maketitle

\begin{abstract}

\textit{Warning:} At the moment this document is a kitchen sink of everything
related to Junta's design. 

Hakell historically lacks workflow-based build tools. Junta is an attempt to
fix that, though it might be able to support other compiled languages as well.

\end{abstract}

\section{Problem Statement}
Junta is a high-level build automation tool that makes use of build workflows.
It is intended to be `a layer above' make-like tools. Junta expects a certain
project structure for the default workflow and extending the workflow by adding
plugin or configuration.

The motivation for this project is that at the moment there are no such build
tool for Haskell, hence all the examples here are concerned with Haskell.
Approach can be potentially extended to support other compiled lanugages, but
that is beyond the scope of the current document.

\section{Design Goals}

\paragraph{Simplicity}
Doing a full build, including testing and documentation using Cabal 1.20
requires significant amount of manual intervention (various commands in
particular order) and some tools to be available ourside the build system
(Haddoc binary, for example). We are looking to break that and allow performing
a full build with a single command.

Also, the tool is intended to incorporate various build activities, from
compilation to publishing the artifacts and that would significantly simplify
development process.

\paragraph{Usability}
Having ma

\paragraph{Economy}

\section{Plugin Architecture}
\subsection{Configuration}
Three different types of targets:
\begin{enumerate}
\item A \textit{phase} is defined by the tool and exposed via API
\item A \textit{goal} is defined by plugin and registered in the API
\item An \textit{execution} is defined in the configuration file, links plugin goals with build phases
\end{enumerate}

\includegraphics[height=200px]{configuration.png}

\section{Build Phases}
TODO add description of the phases.

\subsection{Implementation mechanism}
Main assumption: a phase has at most one prerequisite phase. That way we can
define a phase as being either a special \textit{Init} phase or being a
compsite phase that has one prerequisite.

\textit{Init} phase would read the configuration file, register plugins and set
up links between plugins and executions.

\end{document}

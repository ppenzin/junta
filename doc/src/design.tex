\documentclass[11pt]{article}

\usepackage{amsmath}
\usepackage{graphicx}

\title{Junta\\Build management tool}
\author{Peter Penzin}
\date{\today}

\begin{document}

\maketitle

\begin{abstract}

\textit{Warning:} At the moment this document is a kitchen sink of everything
related to Junta's design. 

Hakell historically lacks workflow-based build tools. Junta is an attempt to
fix that, though it might be able to support other compiled languages as well.

\end{abstract}

\section{Problem Statement}
Junta is a high-level build automation tool that makes use of build workflows.
It is intended to be `a layer above' make-like tools. Junta expects a certain
project structure for the default workflow and extending the workflow by adding
plugin or configuration.

The motivation for this project is that at the moment there are no such build
tool for Haskell, hence all the examples here are concerned with Haskell.
Approach can be potentially extended to support other compiled lanugages, but
that is beyond the scope of the current document.

\section{Design Goals}

\paragraph{Simplicity}
Doing a full build, including testing and documentation using Cabal 1.20
requires significant number of manual steps (various commands in particular
order) and some tools to be available ourside the build system (Haddoc binary,
for example). We are looking to eliminate significant amount of manual
intervention to the point where a full build can be performed with a single
command.

Also, the tool is intended to incorporate various build activities, from
compilation to publishing the artifacts and that would significantly simplify
development process.

\paragraph{Usability}
Usability stems from simplicity, reducing the need for manual management makes build system more usable.

\paragraph{Economy}
Simplifications lead to saving developer's time and therefore resources of the
roganization she works for.

We are also claiming that our choice of implementation language (Haskell) will
help us save our resources while developing this project.

\paragraph{Reliability}
Eliminating unnecessary manual build step would imporve reliability of the
build, by making it more repeatable. 

\section{Plugin Architecture}
Plugin architecture is supposed to make the tool more extensible. There are
three main `actors' involved in defining a build: \textit{the tool itself}
which otlines default build process; \textit{configuration file} which is read
by the tool and which defines plugins and options used for the projects;
\textit{plugins} that perform build actions.

Links between the three provided by three different types of targets:
\begin{enumerate}
\item A \textit{phase} is defined by the tool and exposed via API
\item A \textit{goal} is defined by plugin and registered in the API
\item An \textit{execution} is defined in the configuration file, links plugin goals with build phases
\end{enumerate}

\includegraphics[height=200px]{configuration.png}

\subsection{Configuration}
TODO

\section{Build Phases}
TODO add description of the phases.

\subsection{Implementation mechanism}
Main assumption: a phase has at most one prerequisite phase. That way we can
define a phase as being either a special \textit{Init} phase or being a
compsite phase that has one prerequisite.

\textit{Init} phase would read the configuration file, register plugins and set
up links between plugins and executions.

\end{document}
